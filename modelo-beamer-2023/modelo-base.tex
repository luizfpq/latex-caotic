\documentclass{beamer}
\usepackage{amsfonts,amsmath,oldgerm}
\usepackage[portuguese]{babel}
\DeclareUnicodeCharacter{200B}{I~AM~HERE!!!!}
\usetheme{sintef}

\newcommand{\testcolor}[1]{\colorbox{#1}{\textcolor{#1}{test}}~\texttt{#1}}

\usefonttheme[onlymath]{serif}

\titlebackground*{assets/background}

\newcommand{\hrefcol}[2]{\textcolor{cyan}{\href{#1}{#2}}}

\title{TITULO DA APRESENTAÇÃO}
\subtitle{Subtitulo da apresentação}
\course{Evento|Curso|Turma}
\author{\href{mailto:<user-mail@ifsp.edu.br>}{Nome \textbf{Sobrenome}}}
\IDnumber{1234567}

\begin{document}
\maketitle

\begin{frame}

      This template is a based on \hrefcol{https://www.overleaf.com/latex/templates/sintef-presentation/jhbhdffczpnx}{SINTEF Presentation} from \hrefcol{mailto:federico.zenith@sintef.no}{Federico Zenith} and its derivation \hrefcol{https://github.com/TOB-KNPOB/Beamer-LaTeX-Themes}{Beamer-LaTeX-Themes} from Liu Qilong

      \vspace{\baselineskip}

      In the following you find a brief introduction on how to use \LaTeX\ and the beamer package to prepare slides, based on the one written by \hrefcol{mailto:federico.zenith@sintef.no}{Federico Zenith} for \hrefcol{https://www.overleaf.com/latex/templates/sintef-presentation/jhbhdffczpnx}{SINTEF Presentation}

      % This template is released under \hrefcol{https://creativecommons.org/licenses/by-nc/4.0/legalcode}{Creative Commons CC BY 4.0} license

\end{frame}

\section{Introdução}

\begin{frame}{Beamer para os slides SINTEF}
      \begin{itemize}
            \item Nós assumimos que você sabe usar \LaTeX; se você não sabe,
                  \hrefcol{http://en.wikibooks.org/wiki/LaTeX/}{você pode aprender aqui}
            \item Beamer é uma das classes de documentos mais populares e poderosas para apresentações em \LaTeX
            \item Beamer também possui um detalhado
                  \hrefcol{http://www.ctan.org/tex-archive/macros/latex/contrib/beamer/doc/beameruserguide.pdf}{manual do usuário}
            \item Aqui apresentaremos apenas as funcionalidades mais básicas para que você possa começar rapidamente
      \end{itemize}
\end{frame}

\begin{frame}{Beamer vs. PowerPoint}
      Comparado ao PowerPoint, usar \LaTeX\ é melhor porque:

      \begin{itemize}
            \item Não é o modelo \emph{What-You-See-Is-What-You-Get} (O que você vê é o que você obtém), mas sim \emph{What-You-\emph{Mean}-Is-What-You-Get} (O que você quer dizer é o que você obtém):
                  Você escreve o conteúdo, o computador faz a composição tipográfica.

            \item Produz um \texttt{pdf}: sem problemas com fontes, fórmulas, versões do programa.

            \item É mais fácil manter um estilo consistente, fontes, realces, etc.

            \item A composição matemática no \TeX\ é a melhor:
                  \begin{equation*}
                        \mathrm{i}\,\hslash\frac{\partial}{\partial t} \Psi(\mathbf{r},t) =
                        -\frac{\hslash^2}{2\,m}\nabla^2\Psi(\mathbf{r},t)
                        + V(\mathbf{r})\Psi(\mathbf{r},t)
                  \end{equation*}
      \end{itemize}
\end{frame}

\begin{frame}[fragile]{Primeiros Passos}
      \framesubtitle{Selecionando o Tema SINTEF}
      Para começar a trabalhar com o \texttt{sintefbeamer}, inicie um documento \LaTeX\ com o preâmbulo:
      \begin{block}{Documento Mínimo do Beamer SINTEF}
            \verb|\documentclass{beamer}|\\
            \verb|\usetheme{sintef}|\\
            \verb|\begin{document}|\\
            \verb|\begin{frame}{Olá, mundo!}|\\
            \verb|\end{frame}|\\
            \verb|\end{document}|\\
      \end{block}
\end{frame}

\begin{frame}[fragile]{Página de Título}
      Para configurar uma típica página de título, você deve chamar alguns comandos no preâmbulo:
      \begin{block}{Os Comandos para a Página de Título}
            \begin{verbatim}
      \title{Título de Exemplo}
      \subtitle{Subtítulo de Exemplo}
      \author{Primeiro Autor, Segundo Autor}
      \date{\today}
      \end{verbatim}
      \end{block}

      Você pode então escrever a página de título com o comando \verb|\maketitle|.

      Para definir uma \textbf{imagem de fundo}, use o comando \verb|\titlebackground| antes de \verb|\maketitle|; seu único argumento é o nome (ou caminho) de um arquivo gráfico.

      Se você usar a versão \textbf{com estrela} \verb|\titlebackground*|, a imagem será recortada para exibição na metade direita do slide de título.

\end{frame}

\begin{frame}[fragile]{Escrevendo um Slide Simples}
      \framesubtitle{É realmente fácil!}
      \begin{itemize}[<+->]
            \item Um slide típico contém listas com marcadores.
            \item Essas listas podem ser reveladas sequencialmente.
      \end{itemize}
      \begin{block}{Código para uma Página com uma Lista Enumerada}<+->
            \begin{verbatim}
            \begin{frame}{Escrevendo um Slide Simples}
            \framesubtitle{É realmente fácil!}
            \begin{itemize}[<+->]
            \item Um slide típico contém listas com marcadores.
            \item Essas listas podem ser reveladas sequencialmente.
            \end{itemize}\end{frame}
            \end{verbatim}
      \end{block}
\end{frame}

\section{Personalização}

\footlinecolor{sintefyellow}
\begin{frame}[fragile]{Alterando o Estilo do Slide}
      \begin{itemize}
            \item Você pode selecionar o \textbf{estilo do slide} em branco ou \textit{maincolor} \emph{no preâmbulo} com \verb|\themecolor{white}| (padrão) ou \verb|\themecolor{main}|
                  \begin{itemize}
                        \item Você \emph{não deve} mudar isso dentro do documento: o Beamer não gosta disso.
                        \item Se você \emph{realmente} precisar, talvez seja necessário adicionar \verb|\usebeamercolor[fg]{normal text}| no slide.
                  \end{itemize}
            \item Você pode alterar a \textbf{cor da rodapé} (\emph{footline}) com \verb|\footlinecolor{cor}|
                  \begin{itemize}
                        \item Coloque o comando \emph{antes} de um novo \verb|frame|.
                        \item Existem quatro cores "oficiais": \testcolor{maincolor}, \testcolor{sintefyellow}, \testcolor{sintefgreen}, \testcolor{sintefdarkgreen}, \testcolor{sintefred} (add pelo quirino).
                        \item O padrão é sem rodapé; você pode restaurá-lo com \verb|\footlinecolor{}|.
                        \item Outras cores podem funcionar, mas não há garantias!
                        \item Não deve ser usado com o tema \verb|maincolor|!
                  \end{itemize}
      \end{itemize}
\end{frame}

\begin{frame}[fragile]{Blocos}
      \begin{columns} % adicionando [onlytextwidth], as margens esquerdas serão definidas corretamente
            \begin{column}{0.3\textwidth}
                  \begin{block}{Blocos Padrão}
                        Estes têm uma cor coordenada com a rodapé (e cinza no tema azul)
                        \begin{verbatim}
      \begin{block}{título}
      conteúdo...
      \end{block}
      \end{verbatim}
                  \end{block}
            \end{column}
            \begin{column}{0.7\textwidth}
                  \begin{exampleblock}{Blocos Coloridos}
                        Semelhante aos do lado esquerdo, mas você escolhe a cor. O texto será branco por padrão, mas você pode definir isso com um argumento opcional.
                        \small
                        \begin{verbatim}
      \begin{exampleblock}{título}
      conteúdo...
      \end{exampleblock}
      \end{verbatim}
                  \end{exampleblock}
                  As cores "oficiais" dos blocos coloridos são: \testcolor{sinteflilla}, \testcolor{maincolor}, \testcolor{sintefdarkgreen} e \testcolor{sintefyellow}.
            \end{column}
      \end{columns}
\end{frame}


\footlinecolor{}

\begin{frame}[fragile]{Usando Cores}
      \begin{itemize}[<alert@2>]
            \item Você pode usar cores com o comando \verb|\textcolor{<nome da cor>}{texto}|
            \item As cores são definidas no pacote \texttt{sintefcolor}:
                  \begin{itemize}
                        \item Cores principais: \testcolor{maincolor} e sua variante \testcolor{sintefgrey}
                        \item Três tons de verde: \testcolor{sinteflightgreen}, \testcolor{sintefgreen}, \testcolor{sintefdarkgreen}
                        \item Cores adicionais: \testcolor{sintefyellow}, \testcolor{sintefred}, \testcolor{sinteflilla}
                              \begin{itemize}
                                    \item Estas cores podem ser sombreadas --- consulte a documentação do \verb|sintefcolor| ou o \hrefcol{https://sintef.sharepoint.com/sites/stottetjenester/%
                                                kommunikasjon/grafisk-profil-new/Sider/default.aspx}{manual de perfil da SINTEF} para mais informações.
                              \end{itemize}
                  \end{itemize}
            \item Não \emph{abusar} das cores: \verb|\emph{}| geralmente é suficiente.
            \item Use \verb|\alert{}| para \alert<2->{destacar} algum conteúdo.
            \item<2- | alert@2> Se você destacar demais, o destaque perde o efeito!
      \end{itemize}
\end{frame}



\begin{frame}[fragile]{Adicionando Imagens}
      \begin{columns} % adicionando [onlytextwidth], as margens esquerdas serão definidas corretamente
            \begin{column}{0.7\textwidth}
                  Adicionar imagens funciona como no \LaTeX\ normal:
                  \begin{block}{Código para Adicionar Imagens}
                        \begin{verbatim}
            \usepackage{graphicx}
            % ...
            \includegraphics[width=\textwidth]
            {assets/logo_RGB}
            \end{verbatim}
                  \end{block}
            \end{column}
            \begin{column}{0.3\textwidth}
                  \includegraphics[width=\textwidth]
                  {assets/logo_RGB}
            \end{column}
      \end{columns}
\end{frame}

\begin{frame}[fragile]{Splitting in Columns}
      Dividir a página em colunas é fácil e comum;
      Tipicamente, um lado possui uma imagem e o outro texto:
      \begin{columns}  % adding [onlytextwidth] the left margins will be set correctly
            \begin{column}{0.6\textwidth}
                  Essa é a primeira coluna
            \end{column}
            \begin{column}{0.3\textwidth}
                  E essa é a segunda
            \end{column}
      \end{columns}
      \begin{block}{Código para colunas}
            \begin{verbatim}
\begin{columns}   % adicionando [onlytextwidth], as margens esquerdas serão definidas corretamente
    \begin{column}{0.6\textwidth}
      Essa é a primeira coluna
    \end{column}
    \begin{column}{0.3\textwidth}
      E essa é a segunda
    \end{column}
    % Podem haver mais colunas
\end{columns}
\end{verbatim}
      \end{block}
\end{frame}

\begin{chapter}[assets/background_negative]{}{Slides Especiais}
 \begin{itemize}
       \item Slides de capítulo
       \item Slides com imagem lateral
 \end{itemize}
\end{chapter}

\footlinecolor{sintefred}

\begin{frame}[fragile]{Slides de Capítulo}
      \begin{itemize}
            \item Semelhantes aos slides \verb|frame|, mas com algumas opções adicionais.
            \item Abertos com \verb|\begin{chapter}[<imagem>]{<cor>}{<título>}|.
            \item A imagem é opcional, a cor e o título são obrigatórios.
            \item Existem sete cores "oficiais": \testcolor{maincolor}, \testcolor{sintefdarkgreen}, \testcolor{sintefgreen}, \testcolor{sinteflightgreen}, \testcolor{sintefred}, \testcolor{sintefyellow}, \testcolor{sinteflilla}.
                  \begin{itemize}
                        \item Curiosamente, são \emph{mais} cores oficiais do que as cores da rodapé.
                        \item Ainda assim, pode ser interessante combinar a cor da rodapé com a cor de um slide de capítulo. Fica a seu critério.
                  \end{itemize}
            \item Caso contrário, \verb|chapter| se comporta exatamente como \verb|frame|.
      \end{itemize}
\end{frame}

\begin{sidepic}{assets/background_alternative}{Slides com Imagem Lateral}
      \begin{itemize}
            \item Abertos com \texttt{$\backslash$begin{sidepic}{<imagem>}{<título>}}
            \item Caso contrário, \texttt{sidepic} se comporta exatamente como \texttt{frame}
      \end{itemize}
\end{sidepic}

\footlinecolor{}
\begin{frame}
      \frametitle{Fontes}
      \begin{itemize}
            \item A tarefa primordial das fontes é serem legíveis.
            \item Existem fontes boas...
                  \begin{itemize}
                        \item {\textrm{Use fontes com serifa apenas com projetores de alta definição}}
                        \item {\textsf{Use fontes sem serifa caso contrário (ou se você simplesmente preferir
                                    elas)}}
                  \end{itemize}
            \item ... e fontes nem tão boas assim:
                  \begin{itemize}
                        \item {\texttt{Nunca use fonte monoespaçada para texto normal}}
                        \item {\frakfamily Fontes góticas, caligráficas ou estranhas: devem sempre ser evitadas}
                  \end{itemize}
      \end{itemize}
\end{frame}

\begin{frame}[fragile]{Aparência}
      \begin{itemize}
            \item Para inserir um slide final com o título e agradecimentos finais, use \verb|\backmatter|.
                  \begin{itemize}
                        \item O título também aparece na rodapé junto com o nome do autor, você pode alterar esse texto com \verb|\footlinepayoff|.
                        \item Você pode remover o título do slide final com \verb|\backmatter[notitle]|.
                  \end{itemize}
            \item A proporção de aspecto padrão é 16:9, e você não deve alterá-la para 4:3 em projetores antigos, pois é impossível converter perfeitamente uma apresentação 16:9 para 4:3; os espaçamentos \emph{serão} alterados.
                  \begin{itemize}
                        \item O argumento \texttt{aspectratio} da classe \texttt{beamer} é substituído pelo tema SINTEF.
                        \item Se você \emph{realmente} sabe o que está fazendo, verifique o código do pacote e procure pela classe \texttt{geometry}.
                  \end{itemize}
      \end{itemize}
\end{frame}

\section{Sumário}

\begin{frame}
      \frametitle{Boa Sorte!}
      \begin{itemize}
            \item Isso é o suficiente para uma introdução! Agora você já deve saber o bastante.
            \item Se tiver correções ou sugestões,
                  \hrefcol{mailto:andrea@gasparini.cloud}{envie-as para mim!}
      \end{itemize}
\end{frame}

 % para alterar o que consta aqui, é necessário alterar os parametros em % Define backmatter 
 % no arquivo beamerthemesintef.sty
 % para não mostrar este slide de agradecimento final, é só remover a tag \blackmatter

\backmatter
\end{document}
