\documentclass{beamer}
\usepackage{amsfonts,amsmath,oldgerm}
\usepackage[portuguese]{babel}

\usepackage{graphicx} %Loading the package
\graphicspath{{assets/outsourcing}} %Setting the graphicspath

\DeclareUnicodeCharacter{200B}{I~AM~HERE!!!!}
\usetheme{sintef}

\newcommand{\testcolor}[1]{\colorbox{#1}{\textcolor{#1}{test}}~\texttt{#1}}

\usefonttheme[onlymath]{serif}

\titlebackground*{assets/background}

\newcommand{\hrefcol}[2]{\textcolor{cyan}{\href{#1}{#2}}}

\title{TITULO DA APRESENTAÇÃO}
\subtitle{Subtitulo da apresentação}
\course{Evento|Curso|Turma}
\author{\href{mailto:<user-mail@ifsp.edu.br>}{Nome \textbf{Sobrenome}}}
\IDnumber{1234567}

\begin{document}
\maketitle

\begin{frame}


\frametitle{O que é outsourcing}

\begin{itemize}
\item Contratação de outsourcing é a prática de contratar especialistas externos para trabalhar em projetos específicos.
\item Geralmente é usada para projetos que requerem conhecimento técnico especializado.
\item Pode ser uma opção para projetos que exigem habilidades que não estão disponíveis na equipe interna da empresa.
\end{itemize}

% \begin{figure}[h]
%   \centering
%   \includegraphics[width=0.6\textwidth]{team_image.jpg}
% \end{figure}

\end{frame}


\begin{frame}

\frametitle{Benefícios da contratação de outsourcing}

\begin{itemize}
\item Redução de custos.
\item Acesso a uma equipe de profissionais experientes.
\item Tempo de entrega reduzido.
\item Foco nas competências principais da empresa.
\item Flexibilidade para escalar a equipe com base nas necessidades do projeto.
\end{itemize}

% \begin{figure}[h]
%   \centering
%   \includegraphics[width=0.5\textwidth]{cost_savings_chart.png}
% \end{figure}

\end{frame}







\begin{sidepic}{assets/outsourcing/confused_person.png}{Desafios da contratação de outsourcing}
    \begin{itemize}
    \item Problemas com comunicação e gerenciamento de equipes remotas.
    \item Dificuldades culturais e de idioma.
    \item Dependência de terceiros.
    \item Preocupação com a segurança e proteção de dados.
    \end{itemize}
\end{sidepic}

\begin{frame}

\frametitle{Melhores práticas para contratação de outsourcing}

\begin{itemize}
\item Escolha de um fornecedor confiável e experiente.
\item Estabelecer métricas claras de sucesso.
\item Definir prazos e orçamentos realistas.
\item Ter um gerente de projeto dedicado à coordenação entre a equipe interna e externa.
\item Estabelecer acordos de confidencialidade e segurança de dados.
\end{itemize}

\end{frame}

\begin{frame}

\frametitle{Exemplos de empresas que usam a contratação de outsourcing com sucesso}

\begin{itemize}
\item Microsoft
\item Google
\item Alibaba
\item Coca-Cola
\item Nestlé
\end{itemize}

\end{frame}

\begin{frame}

\frametitle{Conclusão}

\begin{itemize}
\item A contratação de outsourcing pode ser uma opção atrativa para empresas que buscam especialização técnica e redução de custos.
\item É importante considerar os benefícios e desafios antes de fazer a escolha de contratar equipe externa.
\item A escolha de um fornecedor confiável e estabelecimento de acordos claros são fundamentais para o sucesso da contratação de outsourcing.
\end{itemize}

\vspace{1cm}

\centering
\textbf{Obrigado!}

\end{frame}

\backmatter
\end{document}
