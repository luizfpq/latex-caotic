\documentclass[xcolor={table}]{beamer}
%\usetheme{Lapesd}
\usepackage{beamerthemeLapesd}

\usepackage[brazil]{babel}	% coloca as coisas em portugues no sumário.
\usepackage[utf8]{inputenc}
\usepackage[T1]{fontenc}
\usepackage[scaled]{helvet}
\usepackage{amsthm}
\usepackage{ragged2e}
\usepackage{subfig}
\usepackage[table]{xcolor}
\usepackage{ctable}
\usepackage{multicol}
\usepackage{multirow}
\usepackage{fancyvrb}
\usepackage{verbatim}

\definecolor{mapOrange}{rgb}{255,230,204}



\graphicspath{{fig/}}
\DeclareGraphicsExtensions{.pdf,.jpg,.png,.gif}

\def\signed #1{{\leavevmode\unskip\nobreak\hfil\penalty50\hskip2em
  \hbox{}\nobreak\hfil(#1)
  \parfillskip=0pt \finalhyphendemerits=0 \endgraf}}

\renewcommand{\footnotesize}{\tiny}

\newcommand{\Tsup}[1]{%
  \texorpdfstring{\textsuperscript{#1}}{}%
}

\newcommand{\mymacro}[1]{#1}

\addtobeamertemplate{block begin}{}{\justifying}

\captionsetup{labelformat=simple}
\captionsetup[table]{belowskip=0pt}

\title[EVENTO]
{\textbf{Titulo da Apresentação}}

%author[ABREVIATURA]{NOME COMPLETO}
\author[NOME ABREVIADO]{NOME COMPLETO}
\date{30/07/2020}
\institute{SETOR \\ IFSP\\
\\
\url{}}

%================================%================================
%================================%================================
\begin{document}

\begingroup
\makeatletter
\setlength{\hoffset}{-.5\beamer@sidebarwidth}
\makeatother
\begin{frame}[plain,t,noframenumbering]
    
    \titlepage
   
\end{frame}
\endgroup
%================================

\begin{frame}\frametitle{Sumário}
\tableofcontents
\end{frame}

%================================
\section{TITULO}
\begin{frame}[noframenumbering]\frametitle{Sumário}
\tableofcontents[currentsection]
\end{frame}
%\begin{frame}{O que monitorar?}
%\end{frame}
%================================
\subsection{Tópico 1}
\begin{frame}{Tópico 1}
    \item item 1
    \item item 2
\end{frame}
%================================
\subsection{Tópico 2}
\begin{frame}{Tópico 2}
    \item item 1
    \item item 2
\end{frame}
%================================
\subsection{Tópico 3}
\begin{frame}{Tópico 3}
    \item item 1
    \item item 2
\end{frame}
%================================

%================================
\section{EXEMPLO}
\begin{frame}{FIGURA}
\begin{figure}[h!]
    \includegraphics[width=0.4\textwidth, trim=0cm 0cm 0cm 0cm]{fig/Logomarca_IFSP.jpg}
   %\caption{\centering TEXTO Fonte: [1].}
    \end{figure}
\end{frame}
%================================
\section{Duvidas}
\begin{frame}{Perguntas?}
\end{frame}

%================================

%================================
\begin{frame}{Referências}
	\def\newblock{}
	\bibliographystyle{plain}
	\bibliography{referencias}
\end{frame}
%================================

\end{document}